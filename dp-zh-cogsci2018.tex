% Template for Cogsci submission with R Markdown

% Stuff changed from original Markdown PLOS Template
\documentclass[10pt, letterpaper]{article}

\usepackage{cogsci}
\usepackage{pslatex}
\usepackage{float}
\usepackage{caption}

% amsmath package, useful for mathematical formulas
\usepackage{amsmath}

% amssymb package, useful for mathematical symbols
\usepackage{amssymb}

% hyperref package, useful for hyperlinks
\usepackage{hyperref}

% graphicx package, useful for including eps and pdf graphics
% include graphics with the command \includegraphics
\usepackage{graphicx}

% Sweave(-like)
\usepackage{fancyvrb}
\DefineVerbatimEnvironment{Sinput}{Verbatim}{fontshape=sl}
\DefineVerbatimEnvironment{Soutput}{Verbatim}{}
\DefineVerbatimEnvironment{Scode}{Verbatim}{fontshape=sl}
\newenvironment{Schunk}{}{}
\DefineVerbatimEnvironment{Code}{Verbatim}{}
\DefineVerbatimEnvironment{CodeInput}{Verbatim}{fontshape=sl}
\DefineVerbatimEnvironment{CodeOutput}{Verbatim}{}
\newenvironment{CodeChunk}{}{}

% cite package, to clean up citations in the main text. Do not remove.
\usepackage{cite}

\usepackage{color}

% Use doublespacing - comment out for single spacing
%\usepackage{setspace}
%\doublespacing


% % Text layout
% \topmargin 0.0cm
% \oddsidemargin 0.5cm
% \evensidemargin 0.5cm
% \textwidth 16cm
% \textheight 21cm

\title{Expectations bias judgments of harm against others}


\author{{\large \bf Derek Powell} \\ \texttt{derekpowell@stanford.edu} \\ Department of Psychology \\ Stanford University \And {\large \bf Zachary Horne} \\ \texttt{zachary.horne@asu.edu} \\ Department of Psychology \\ Arizona State University}

\begin{document}

\maketitle

\begin{abstract}
People's expectations play an important role in their evaluations and
reactions to events. There is often disappointment when events fail to
meet expectations---sometimes even when the events are still positive
overall---and there is a special thrill to having one's expectations
exceeded. In four studies, we examined how expectations influence
people's judgments of events where another person or people were harmed.
Participants judged pairs of events where a victim experienced a similar
harm, but where victims were at different prior risk of being harmed. We
found that people judged these events as being worse when they were less
expected--that is, when the victims were initially at lower risk of
being harmed. We argue that this bias has pernicious moral consequences.

\textbf{Keywords:}
Judgment and decision-making; Moral judgments; Bias
\end{abstract}

\section{Introduction}\label{introduction}

On the evening of November 13th, 2015, a terrorist attack in Paris left
130 people dead and injured over 300 more. In the aftermath, millions
took to Twitter to express their shock, horror, and outrage at this
tragedy under hashtags like \texttt{\#parisattacks} and
\texttt{\#jesuisparis}. Yet, most of those mourning had little to say 15
hours earlier, when another tragic attack killed at least 43 people in
Beirut. Several factors are surely at play in these different reactions
(e.g.~group affiliations, Brewer, 1999), yet one potentially fundamental
factor has gone unmentioned: the fact that the Paris attack was more
surprising than the attack in Beirut. In contrast to France, Lebanon had
experienced dozens of terrorist bombings and attacks in recent years.
Consequently, Beirut may seem to many like the sort of place where
``these things happen,'' whereas Paris is perceived as being stable and
safe.

We can see in everyday experience that people's evaluations of events
often depend on their expectations about those events. There is often
disappointment when events fail to meet expectations, and there is a
special thrill to having one's expectations exceeded. Anecdotally, these
forces seem to drive people's tendencies to root for the underdog, hold
surprise parties, and foreshadow bad news to ease its delivery (Bell,
1985). Indeed, laboratory studies suggest that expectations play an
important role in people's evaluations of the utility of an event.
Mellers and colleagues (1997) found that expectations influenced
affective reactions during a gambling task. Given a gamble with a 10\%
chance to win \$30 and 90\% chance to win \$0, little disappointment is
felt upon resolving with the \$0 outcome, but considerable elation is
experienced upon winning the \$30. Conversely, given a gamble with a
90\% chance of winning \$60 and 10\% chance of winning \$0, the \$0
outcome engenders considerable disappointment and the \$60 relatively
muted enjoyment. In fact, in gambles similar to these, Mellers and
colleagues (1997) found that people were happier with the smaller
unexpected gain than with the larger but more expected gain (also see
Shepperd \& Mcnulty, 2002).

Considering the important roles of both utility and affect in moral
judgment (e.g., Greene, Sommerville, Nystrom, Darley, \& Cohen, 2001),
it is plausible that expectations might shape how people react to
morally harmful events, such as acts of terrorism. However, unlike in
the context of gambles, in these contexts the effects of expectations on
evaluations may have harmful consequences. When events are shocking,
people may perceive them as more severe and consequently be roused to
action. In contrast, when events harm victims who are generally
considered to be at greater risk--the poor, sick, or those living in
unstable regions of the world, reactions may be more muted. If so,
observers who learn of these events may experience reduced moral
concern, and thus be less likely to donate time or money to aid victims,
to take political action, and so forth.

Here, we examined whether people's evaluations of morally harmful events
are affected by their expectations about those events. We asked people
to compare pairs of simple events where a victim suffered an identical
harm, but where the events differed in their prior probability. For each
pair of events, participants were asked to judge which event was worse.
Across four studies, we found that people tended to view unexpected
negative events as worse, even when the harm to victims was
identical.\footnote{A minor caveat is in order. We conducted an
  additional fifth study that did not exhibit the predicted effect of
  expectations on evaluations. We present and discuss this study in our
  supplemental online materials. We have omitted discussion of this
  study in the main text because the items we created in this study did
  not cleanly separate the unexpectedness of an event with how much harm
  (objectively) the event entailed. For example, one item used in this
  study was as follows: ``A window washer is killed instantly when he
  falls from the 2nd floor after a cable snaps'' vs. ``A window washer
  is killed instantly when he falls from the 10th floor after a cable
  snaps''. Though it may be more surprising to be killed after a
  two-story fall, it may simply be worse to fall ten stories. Despite
  these concerns, we have made this data available online at
  \url{https://osf.io/a6pbj/}.}

\section{General Methods}\label{general-methods}

Here we present four studies examining the role of expectations in moral
evaluations. Studies 1a and 1b are initial studies demonstrating the
hypothesized phenomena: stronger reactions toward unexpected as compared
to expected negative events. Studies 2a and 2b are refinements on these
original studies, intended to further demonstrate the generalizability
of results and to institute some methodological improvements.

\subsection{Materials}\label{materials}

In all studies, participants were presented with a series of trials
where they read brief (one sentence) descriptions of two different
events and were asked to indicate which of the two events seemed worse.
In ``experimental'' trials, the two events were highly similar, but
differed in their prior probabilities: one event was more expected and
one more unexpected. These prior expectations were manipulated by
changing the context in which the events occurred. For example,
participants considered the following stimulus:

\begin{itemize}
\item
  ``A 30 year old man in California dies in an earthquake''
  {[}Expected{]}
\item
  ``A 30 year old man in Oklahoma dies in an earthquake''
  {[}Unexpected{]}
\end{itemize}

In each event, the harm to the victim is the same (here, death) but one
event is should be more expected than the other, given the different
likelihoods of earthquakes occurring in California versus Oklahoma.

Each study contained between 6 and 12 experimental event-pairs that
spanned a variety of different events and contexts. The events described
were either the result of natural forces and misfortune (1a and 2a) or
the result of an action by another person (1b and 2b). All experimental
materials for these studies are available as supplemental online
materials at \url{https://osf.io/a6pbj/}.

Studies 1a and 1b also included ``equivalent'' filler trials. In these
trials, the two events differed in trivial contextual details that we
did not expect would affect participants' judgments. For example:

\begin{itemize}
\item
  ``A man in Connecticut starts a house fire.'' {[}Equally expected{]}
\item
  ``A man in New Hampshire starts a house fire.'' {[}Equally expected{]}
\end{itemize}

These filler trials were meant to prevent participants from becoming
explicitly aware of the structure of the experimental trials.

In addition to these filler trials, studies 2a and 2b added
``non-equivalent'' filler trials, the two events differed substantially
in the degree of harm suffered by a victim, so that one event was
expected to be seen as considerably worse than the other. For example:

\begin{itemize}
\item
  ``An 11-year-old child sets a doll on fire'' {[}More severe{]}
\item
  ``A 12-year-old child sets a cat on fire'' {[}Less severe{]}
\end{itemize}

These trials were included to allow participants a chance to use the
extremes of the response scale and to reduce any task demands that might
drive them to make artificially fine-grained distinctions between the
severity of events.

\subsection{Exclusions}\label{exclusions}

Each study also made use of attention-check items. These questions asked
participants to enter a particular response to ensure that they were
paying attention and reading the items as they proceeded through the
study. A final question asked participants if they had paid attention
and taken the study seriously, encouraging them to be honest in their
replies.

\subsection{Data Analysis}\label{data-analysis}

We analyzed our data by performing Bayesian estimation using the
probabilistic programming language Stan (Carpenter et al., 2017). We
tested our predictions by computing Bayes Factors (i.e.~BF01) on the
intercept term of our regression model using the hypothesis function in
the R package brms. As a reminder, Bayes Factors express the ratio of
the probability of data under the null hypothesis to the probability of
the data under an alternative hypothesis. Therefore, larger Bayes
Factors indicate that the data are more likely under the null hypothesis
(e.g., that the intercept is not different from zero) than the
alternative hypothesis (e.g., that the intercept is different from 0),
and vice versa. Bayes Factors can be influenced by prior choices so we
also performed prior robustness checks to confirm that the prior alone
was not accounting for the effects that we predicted.

\section{Study 1a}\label{study-1a}

\subsection{Participants}\label{participants}

A total of 55 participants were recruited from Amazon's Mechanical Turk
work distribution website (mTurk). Of these, 53 passed attention checks
and were included in the final analyses (24 male, 29 female, median age
= 32 years old). All participants were paid \$1.00 for their
participation.

\subsection{Materials and procedure}\label{materials-and-procedure}

Participants judged 12 experimental event-pairs and 12 equivalent filler
event-pairs. The events were described in the passive voice, and
participants were asked to judge which event seemed worse. For example:

\begin{itemize}
\item
  A 32 year old woman gets food poisoning after eating a hamburger at a
  fast food restaurant. {[}Expected{]}
\item
  A 32 year old woman gets food poisoning after eating a hamburger at a
  four star restaurant. {[}Unexpected{]}
\end{itemize}

On each trial, participants were presented with the event-pair stimulus
and had to judge which outcome was worse in a two-alternative forced
choice task. The two events were labeled ``Outcome 1'' and ``Outcome 2''
and their order was randomized.

\subsection{Results and discussion}\label{results-and-discussion}

We predicted that people would think that events where unexpected harm
occurred were worse than events that entail similar harm but were
comparatively more expected. As indicated in Figure 1, when forced to
choose, people judged that unexpected negative events were worse than
expected events. To confirm this difference formally, we fit a Bayesian
logistic random effects model with participants' responses as the
dependent variable (1 = unexpected event is worse; 0 = expected event is
worse) and a random intercept for subject. The intercept in this model
represents the log-odds of selecting the unexpected event as being
worse. Thus, by examining the population-level intercept, we can test
whether participants were biased toward selecting the unexpected event
(b \textgreater{} 1), the expected event (B \textless{} 1), or were
unbiased (B = 1). Consistent with our hypothesis, we found that people
were much more likely to think that unexpected events were worse than
events that were expected, Intercept = 1.141, 95\% CI {[}0.916,
1.386{]}, BF01 \textless{} .001. Bayes factors and the estimate of the
intercept were similar under different prior choices.

Figure 2 (panel 1) shows participants' responses broken-down by
individual items. Participants' bias toward selecting the unexpected
event as worse was largely consistent across the 12 experimental items.

\begin{CodeChunk}
\begin{figure}[H]

{\centering \includegraphics{figs/fig1-1} 

}

\caption[Proportion of response choices across studies 1-4 (pooled across items)]{Proportion of response choices across studies 1-4 (pooled across items). Error bars indicate 95 \% bootstrapped CI.}\label{fig:fig1}
\end{figure}
\end{CodeChunk}

\section{Study 1b}\label{study-1b}

Consistent with prior work suggesting that people's utility evaluations
are affected by their expectations (e.g., Mellers et al., 1999), Study
1a provided evidence that people view unexpected moral harm as worse
than expected moral harm. To expand on these findings, and conceptually
replicate the results of Study 1a, in Study 1b we asked people to
evaluate other people's actions rather than the outcomes of events.

This study allowed us to confirm that, among other things, minor
differences in the wording of our stimuli were not responsible for the
effect observed in Study 1a.

\subsection{Participants}\label{participants-1}

A total of 112 participants were recruited from Amazon's Mechanical Turk
work distribution website (mTurk). Of these, 110 passed attention checks
and were included in the final analyses (61 male, 49 female, median age
= 30 years old). All participants were paid \$1.00 for their
participation.

Small sample sizes tend to overestimate effect sizes (Button et al.,
2013). Consequently, we also increased our sample size to confirm that
the large effect observed in Study 1a was actually reflective of the
effect of expectations on evaluations of moral harm.

\subsection{Materials and procedure}\label{materials-and-procedure-1}

Participants judged 6 experimental event-pairs and 6 equivalent filler
event-pairs. These event-pairs were adapted from event-pairs in Studies
1a and 1b in which a victim is harmed by another person's actions. The
events were rephrased into the active voice in order to focus on the
agent who took the action rather than the victim who was harmed by it.
The actions participants selected between were labeled ``Action 1'' and
``Action 2.'' For example, participants were presented with the
following stimulus and had to judge which action was worse:

\begin{itemize}
\item
  ``A wanted criminal shoots and wounds a police officer during a drug
  raid.'' {[}Expected{]}
\item
  ``A wanted criminal shoots and wounds a police officer during a
  traffic stop.'' {[}Unexpected{]}
\end{itemize}

As in experiment 1a, participants were asked to choose which of the two
actions seemed worse in a two-alternative forced choice task.

\subsection{Results and discussion}\label{results-and-discussion-1}

We predicted that people would think that unexpected actions that caused
harm were worse than expected actions that entailed similar harms. Just
as we found in Study 1a, people judged that unexpected actions were
worse than expected actions (see Figure 1, panel 2). We confirmed this
difference formally by again fitting a Bayesian logistic random effects
model with participants' responses as the dependent variable (1 =
unexpected action is worse; 0 = expected action is worse) and a random
intercept for subject. This analysis indicated that people were more
likely to think that actions that were unexpected were worse than
actions that were expected, Intercept = 0.675, 95\% CI {[}0.493,
0.862{]}, BF01 \textless{} .001.

Figure 2 shows participants' responses broken-down by individual items.
Participants' bias toward selecting the unexpected action as worse were
reasonably consistent across the six experimental items but there
appeared to be more variation than we observed in Study 1a.

In summary, Study 1b suggests that people think that unexpected actions
are worse than expected actions, again indicating that when comparing to
events, people's reactions to negative events are influenced by their
expectations.

\begin{CodeChunk}
\begin{figure*}[h]

{\centering \includegraphics{figs/fig2-2col-1} 

}

\caption[Responses by item for studies 1-4]{Responses by item for studies 1-4. Error bars indicate standard errors. Responses in studies 2a and 2b are represented using scale-means for visualization purposes only (higher scores indicate greater bias toward unexpected event).}\label{fig:fig2-2col}
\end{figure*}
\end{CodeChunk}

\section{Study 2a}\label{study-2a}

In Studies 1a and 1b we found that people's judgments of events were
biased by their expectations about those events. When forced to choose
between two events, participants decided that unexpected events were
worse than expected events. In Study 2a, we sought to test our
hypothesis using a more conservative method. Accordingly we made two
changes in Study 2a: First, we introduced a new type of filler item
``non-equivalent'' filler trials and 2) we provided participants with a
more expressive response scale so that if they viewed the events under
consideration as equally harmful, their responses could reflect their
attitude.

\subsection{Participants}\label{participants-2}

A total of 108 participants were recruited from Amazon's Mechanical Turk
work distribution website (mTurk). Of these, 103 passed attention checks
and were included in the final analyses (59 male, 44 female, median age
= 31 years old). All participants were paid \$1.00 for their
participation.

\subsection{Materials and procedure}\label{materials-and-procedure-2}

Participants judged 12 experimental event-pairs. In all of these
event-pairs, a victim suffers a negative outcome due to misfortune,
rather than another person's actions. Some event-pairs were reused from
Study 1a without modification, others were revised or novel to improve
the generalizability of our findings. These materials were created by
(1) eliminating materials that may have confounded expectations with,
for instance, an out-group bias and (2) creating novel items to again
increase the generalizability of our findings. See supplemental online
materials for a full list of items used in each study
\url{https://osf.io/a6pbj/}.

In addition, participants judged 12 filler event-pairs. We introduced a
new type of filler event-pair: ``non-equivalent'' event-pairs. As
described previously, these are event-pairs that clearly differ in the
degree of harm suffered or committed. For example, participants were
presented with this stimulus and had to judge which was worse:

\begin{itemize}
\item
  ``A man in Washington carjacks someone at gunpoint.'' {[}More severe
  action{]}
\item
  ``A man in Oregon steals a parked car.'' {[}Less severe action{]}
\end{itemize}

These items were introduced to address the concern that the high
similarity within all event-pairs may drive participants to make
overly-fine distinctions in their judgments. Such a task demand might
inflate the effect sizes we observed in Studies 1a and 1b. Of the 12
filler events, six were ``equivalent'' event-pairs like those used
previous studies and six were non-equivalent event pairs.

As in Study 1a, on each trial of the study, participants were presented
with a pair of actions labeled ``Outcome 1'' and ``Outcome 2'' and were
asked, ``Which outcome seems worse?'' However, unlike previous studies,
in Study 2a participants made their rating on a five-point scale
(Outcome 1 seems worse, Outcome 1 seems a little worse, neither seems
worse, Outcome 2 seems a little worse, Outcome 2 seems worse). By
forcing a choice between the two events, experiments 1a and 1b may have
inflated the degree of bias participants exhibited. This more expressive
response-scale was used in experiments 2a and 2b to avoid this concern.

\subsection{Results and discussion}\label{results-and-discussion-2}

The events participants were asked to compare were, by design, highly
similar. Consequently, we expected that participants' would typically
indicate that neither event seemed worse. This was by far the choice
participants most frequently made (see Figure 1, panel 3). However, we
also observed that when participants did perceive one event was worse
than the other, they were biased to perceive unexpected negative events
as worse than more-expected negative events.

To examine these findings formally, we performed cumulative (ordinal)
regression using a Bayesian random effects model with participants'
scale responses as the dependent variable (1 to 5) and a random
intercept for subject. This model produces four intercept coefficients,
representing the cumulative log-odds of responses at each scale point or
higher. For instance, the second coefficient represents the log-odds
participants chose a 2 (``outcome 2 seems slightly worse'') or lower on
the scale. Similarly, the third intercept coefficient represents the
log-odds participants chose a 3 or lower on the scale. By comparing the
second intercept coefficient to the inverse of the third intercept
coefficient (thereby representing the log-odds \emph{not} choosing a 3
or lower--i.e., choosing a 4 or 5), we can test whether participants
were more likely to choose the expected or unexpected event as being
worse in cases where they did not choose the neither option. This
analysis indicated that people were more likely to think that events
that were unexpected were worse than events that were expected, BF01
\textless{} .001 (see supplemental online materials for full model
results) -- when participants did exhibit a bias in their responses
about which event was worse, they reliably chose the unexpected event
was worse than the expected event.

However, these findings should be qualified by acknowledging the
considerable inter-item variability across the 12 items. Figure 2, panel
3 shows participants' responses across individual items. For
visualization purposes only, we display these results using the mean
response across the 5-point scale. Participants were strongly biased to
perceive the unexpected event as worse for approximately half of the
items, but were less strongly-biased for others, and slightly biased in
the reverse direction for two items.

\section{Study 2b}\label{study-2b}

\subsection{Participants}\label{participants-3}

A total of 114 participants were recruited from Amazon's Mechanical Turk
work distribution website (mTurk). Of these, 106 passed attention checks
and were included in the final analyses (48 male, 58 female, median age
= 31 years old). All participants were paid \$1.00 for their
participation.

\subsection{Materials and procedure}\label{materials-and-procedure-3}

Participants judged ten experimental event-pairs, five ``equivalent''
filler event-pairs, and five ``non-equivalent'' filler event-pairs. We
created additional items in this study to improve and expand upon the
event-pairs used in Study 1b.

As in Study 1b, these events all involved an action that harmed a
victim. On each trial of the study, participants were presented with a
pair of actions labeled ``Action 1'' and ``Action 2'' and were asked,
``Which action seems worse?'' Using the same procedure as Study 2a,
participants made their rating on a five-point scale (Action 1 seems
worse, Action 1 seems a little worse, neither seems worse, Action 2
seems a little worse, Action 2 seems worse).

\subsection{Results and discussion}\label{results-and-discussion-3}

Participants pattern of responses were similar to those observed in
Study 2a. We found that participants chose the ``neither'' option in the
majority of trials, but when participants did perceive one action as
worse than the other, they were biased to perceive unexpected negative
actions as worse than more-expected negative actions (Figure 1, panel
4). To examine these findings formally, we again performed cumulative
(ordinal) regression using a Bayesian random effects model with
participants' scale responses as the dependent variable (1 to 5) and a
random intercept for subject. To test our hypothesis, we compared the
Bayes Factor for intercept coefficients representing the log-odds of
choosing the expected and unexpected actions as worse or slightly worse.
As predicted and suggested by Figure 1, this analysis indicated that
people were more likely to think that actions that were unexpected were
worse than actions that were expected, BF01 \textless{} .001 (see
supplemental online materials for full model results).

Here too, our findings should be qualified by acknowledging the
considerable variability across the 10 items of Study 2b (see Figure 2,
panel 4). As shown in the plot, participants were strongly biased to
perceive the unexpected event as worse for four of the items, but showed
almost no bias for the other six items.

\section{Discussion}\label{discussion}

The results of four studies suggest that people view unexpected harmful
events more negatively than expected harmful events. Just as people
react more strongly to unexpected monetary gains and losses (Mellers et
al., 1997), people similarly react more severely to unexpected moral
harm than expected moral harm--judging those unexpected events as
``worse''.

Why should our expectations influence our reactions to events? A number
of researchers have sought to develop theories of disappointment--the
psychological reactions that result when experiences fail to meet
expectations--and its role in evaluation and decision-making (e.g.,
Bell, 1985; Gul, 1991; Loomes \& Sugden, 1986). These theories posit
that decisions and evaluations are affected by the objective (e.g.,
economic) utilities of options and events, as well as disappointment
individual people experience as a function of their expectations.
Alternately, numerous theories of decision-making, including Prospect
Theory (Kahneman \& Tversky, 1979; Tversky \& Kahneman, 1992), have
emphasized the role of relative comparisons in evaluation and
decision-making. In this vein, expectations might help set the reference
points against which people compare potential future outcomes.

On these accounts, the influence of expectations on evaluation is simply
a human quirk, a result of the way we evaluate events and decisions. In
contrast, we suspect that expectations may influence evaluation through
more principled means. The surprise of unexpected events may seem
irrelevant to moral evaluations, but it is vital to learning. In
Information Theory, the information carried by an event is a direct
function of its prior probability, such that low probability events
carry more information than high probability events (Shannon, 1948).
Likewise, the violation of expectations has long been recognized as
fundamental to associative and animal learning models (e.g., Rescorla \&
Wagner, 1972). We suggest that people learn more about the state of the
world when their expectations are violated by shocking world events, as
compared to when they are affirmed by less surprising events. In this
light, it seems intuitive that people would have stronger reactions to
those surprising events. Still, the consequence of this dynamic is
apparently suboptimal moral behavior.

\subsection{Limitations}\label{limitations}

Although we consistently observed a bias to judge unexpected events and
actions as worse than expected events across four studies, in studies 2a
and 2b, we also observed that the extent of the bias was quite dependent
on the specific content of the items. This is perhaps an unsurprising
consequence of our decision to use relatively naturalistic items and to
manipulate expectations about these events implicitly by manipulating
the context in which those events occurred. This technique has the
obvious virtue of affording these items some degree of realism (as
compared to artificial gambling tasks using explicitly stated
probabilities that participants may or may not believe), but
manipulating context may affect other aspects of participant's
interpretation of these actions, potentially introducing confounds. We
sought to guard against this possibility by including a variety of
different items and contextual manipulations. Still, future research is
needed both to broaden these findings and to establish converging
evidence through methods that are not subject to these concerns.

\subsection{Conclusions}\label{conclusions}

The bias to view unexpected harm as worse than more expected harm
threatens to impose a vicious and morally pernicious cycle: For
instance, people living in geo-politically unstable regions or in the
developing world are often those who are most affected by terrorism,
famine, and natural disasters, and are the very people in greatest need
of assistance and concern from the world at-large. However, for these
very reasons, it is often unsurprising when harm befalls people living
in these circumstances. Our findings suggest a bias whereby the people
most likely to suffer and be victimized are the very people for whom
others are least likely to be moved to help. Future research should aim
to understand the processes by which this bias arises and to identify
how it might be counteracted.

\section{Acknowledgements}\label{acknowledgements}

We would like to acknowledge the help of Keith Holyoak, Alan Fiske,
Matthew Lieberman, Hongjing Lu, John Hummel, and Ellen Markman for their
comments and support for the project.

\section{References}\label{references}

\setlength{\parindent}{-0.1in} \setlength{\leftskip}{0.125in} \noindent

\hypertarget{refs}{}
\hypertarget{ref-Bell1985}{}
Bell, D. (1985). Disappointment In Decision Making Under Uncertainty.
\emph{Operations Research}, \emph{33}(1), 1--27.
\url{http://doi.org/10.1287/opre.33.1.1}

\hypertarget{ref-Brewer1999}{}
Brewer, M. B. (1999). The psychology of prejudice: Ingroup love or
outgroup hate? \emph{Journal of Social Issues}, \emph{55}(3), 429--444.
\url{http://doi.org/10.1111/0022-4537.00126}

\hypertarget{ref-Button2013}{}
Button, K. S., Ioannidis, J. P. A., Mokrysz, C., Nosek, B. A., Flint,
J., Robinson, E. S. J., \& Munafò, M. R. (2013). Power failure: why
small sample size undermines the reliability of neuroscience.
\emph{Nature Reviews. Neuroscience}, \emph{14}(5), 365--76.
\url{http://doi.org/10.1038/nrn3475}

\hypertarget{ref-Carpenter2017}{}
Carpenter, B., Gelman, A., Hoffman, M. D., Lee, D., Goodrich, B.,
Betancourt, M., \ldots{} Riddell, A. (2017). Stan: A probabilistic
programming language. \emph{Journal of Statistical Software},
\emph{76}(1). \url{http://doi.org/10.18637/jss.v076.i01}

\hypertarget{ref-Greene2001}{}
Greene, J. D., Sommerville, R. B., Nystrom, L. E., Darley, J. M., \&
Cohen, J. D. (2001). An fMRI investigation of emotional engagement in
moral judgment. \emph{Science (New York, N.Y.)}, \emph{293}(5537),
2105--8. \url{http://doi.org/10.1126/science.1062872}

\hypertarget{ref-Gul1991}{}
Gul, F. (1991). A theory of disappointment aversion.
\emph{Econometrica}, \emph{59}(3), 667--686.
\url{http://doi.org/10.2307/2938223}

\hypertarget{ref-Kahneman1979}{}
Kahneman, D., \& Tversky, A. (1979). Prospect Theory: An Analysis of
Decision under Risk. \emph{Econometrica}, \emph{47}(2), 263--292.

\hypertarget{ref-Loomes1986}{}
Loomes, G., \& Sugden, R. (1986). Disappointment and Dynamic Consistency
in Choice under Uncertainty. \emph{Review of Economic Studies},
\emph{53}(2), 271--282. \url{http://doi.org/10.2307/2297651}

\hypertarget{ref-Mellers1997}{}
Mellers, B. a, Schwartz, a., Ho, K., \& Ritov, I. (1997). Decision
Affect Theory: Emotional Reactions to the Outcomes of Risky Options.
\emph{Psychological Science}, \emph{8}(6), 423--429.
\url{http://doi.org/10.1111/j.1467-9280.1997.tb00455.x}

\hypertarget{ref-Rescorla1972}{}
Rescorla, R. A., \& Wagner, A. R. (1972). A theory of Pavlovian
conditioning: Variations in the effectiveness of reinforcement and
nonreinforcement. In \emph{Classical conditioning ii current research
and theory} (Vol. 21, pp. 64--99).
\url{http://doi.org/10.1101/gr.110528.110}

\hypertarget{ref-Shannon1948}{}
Shannon, C. E. (1948). A Mathematical Theory of Communication.
\emph{Bell System Technical Journal}, \emph{27}(3), 379--423.
\url{http://doi.org/10.1002/j.1538-7305.1948.tb01338.x}

\hypertarget{ref-Shepperd2002}{}
Shepperd, J. a, \& Mcnulty, J. K. (2002). The affective consequences of
expected and unexpected outcomes. \emph{Psychological Science},
\emph{13}(1), 85--88. \url{http://doi.org/10.1111/1467-9280.00416}

\hypertarget{ref-Tversky1992}{}
Tversky, A., \& Kahneman, D. (1992). Advances in prospect theory:
Cumulative representation of uncertainty. \emph{Journal of Risk and
Uncertainty}, \emph{5}, 297--323.

\end{document}
